\section{Introduction}

\section{Présentation de l'Organisme d'Accueil}

FY COMPUTING, est  une entreprise créée en Janvier 2013, basée à Rabat et spécialisée dans l’Informatique de pointe et le Smart COMPUTING, lancée par Emy Capital, holding d’investissement de  Mr. Yahya EL MIR, ex-président et directeur général du groupe SQLI, et présidée par  Mr. Mohammed Feidi BOUZAGBAH, co-fondateur et directeur général. 
\newline

La spécialisation de FY COMPUTING dans un tel domaine n’est pas anodine puisque l’arrivée de l’internet a progressivement développé une troisième ère de l’Informatique qui est « l’Informatique pour le grand public ».          
\newline

La  société propose des nouvelles technologies et des services de haut niveau destinés, aux entreprises et au grand public. Parmi ces solutions on trouve :

\begin{itemize}
\item \textbf{iWay :} Digital Vision from Strategy to Execution Governance \& Mentoring  Monitor.
\item \textbf{iWalk :} Digital IT Smart Project Catalog Services platform Express Rollout.
\item \textbf{iWheel :} C'est un modèle pour piloter la transformation digitale. Il est conçu sur trois principes :

\begin{itemize}
\item Une démarche apprenante qui permet aux entreprises de tester rapidement de nouveaux concepts et d’en tirer les enseignements.

\item Un modèle de maturité digitale basé sur les principes du « Total Quality Management ». Les entreprises peuvent ainsi faire évoluer leur culture digitale et leur organisation de manière structurée et progressive.

\item Les meilleurs outils et pratiques générés par la richesse des écosystèmes afin de gagner du temps et accélérer les capacités digitales.
\end{itemize}
\end{itemize}

\subsection{Les Valeurs de FYComputing}
Parmi les valeurs de l’entreprise FY COMPUTING on trouve : 

\begin{itemize}
\item \textbf{Le goût de l’innovation :}

L’entreprise cherche en permanence à offrir des produits/services uniques sur le marché et dont les clients ne pourront pas s’en passer.

\item \textbf{La culture du partenariat :}

Pour innover, aller vite, satisfaire ses clients finaux, obtenir des résultats, un écosystème de bons partenaires fiables et pointus dans leur domaines est indispensable. Cela oblige à avoir des partenaires qui naturellement ont le goût de l’excellence et du travail bien fait.

\item \textbf{Santé financière :}

FY COMPUTING considère la santé financière d’une entreprise comme la santé physique d’une personne. Vous ne pouvez réellement entreprendre et aller au bout de vos idées sans une bonne santé. C’est la raison pour laquelle elle est dotée d’un positionnement basé sur l’excellence et d’un business model solide, garantissant la bonne santé financière de l’entreprise et son indépendance.
\end{itemize}


\subsection{Positionnement de FY COMPUTING}
Le digital transforme la façon de produire, de vendre et de communiquer. La révolution digitale impose de nouveaux usages impactant les entreprises, leur modèle, leur management, l’ensemble de leurs métiers, et cela quel que soit le secteur d’activité. Les règles du jeu changent, l’IT doit s’adapter. Les entreprises ont besoin de partenaire agile capable de les alimenter en innovation et de répondre aux nouveaux usages et attentes de leurs clients. Le positionnement de FY COMPUTING est de répondre à ces nouvelles exigences.
\newline
\newline
Simplifier l’informatique pour les clients dans des domaines stratégiques très complexes telle pourrait être résumée la mission de FY COMPUTING. Ceci définit ses domaines d’activités comme suit :

\subsubsection{La Transformation Digitale}
L’accélération des mutations technologiques et l’évolution rapide des usages des outils informatiques marquent le marché marocain aujourd'hui, la technologie s’immisce aujourd’hui à travers l’ensemble de ces silos, dans toutes les couches de l’entreprise, jusqu’au cœur de la relation client. En rapprochant les fonctions, quand elle ne permet pas de les fusionner, la technologie est un levier saisissant d’amélioration des processus. Elle ouvre de nouvelles opportunités pour l’organisation. Ainsi la digitalisation des processus de transformation non seulement le business model (le développement de nouveaux produits, nouveaux services numériques ou objets connectés ; Diversification dans les activités numériques), les services et les opérations (Développement de nouveaux canaux d’interaction, Personnalisation de la relation client, Digitalisation des processus et des opérations terrain) , mais également le working model (Digitalisation des processus internes, Adaptation des organisations au travers de nouveaux modèles de travail et des systèmes de management, Développement de compétences numériques).
\newline
\newline
Intelligence stratégique, excellence opérationnelle et connaissance approfondie, sont les clés de la réussite d’une transformation digitale. En revanche, il faut piloter cette transformation avec différents modèles de gouvernance en s’adaptant à la maturité de l’entreprise. C’est une approche coordonnée qui doit être alignée avec une vision et une stratégie. C’est pour cela qu’elle doit se faire avec l’appui de la direction générale et du département informatique, qui va standardiser tous les processus. Cette transformation globale de l’entreprise doit se faire de façon cohérente par rapport aux métiers. Une bonne transformation digitale touche les clients, les services, les fournisseurs, etc. C’est comme une économie intégrée.
\newline
\newline
Inscrites dans cette ère, FY COMPUTING en collaboration avec iRévolution ont lancé iWheel, modèle de maturité digitale permettant aux entreprises de piloter leur transformation digitales.

\subsubsection{Smart Computing}
Le Smart Computing est une informatique plus simple, plus amusante, plus fiable qui a permis à de petites entreprises de devenir des géants en seulement une décennie. Il peut être défini comme l'état d'un système d'information qui permet de fournir, au moment approprié ou dans un contexte donné, l'ensemble des informations (internes ou externes), leur synthèse et les analyses utiles, afin que son utilisateur prennent les décisions pertinentes, et entreprennent les actions utiles, à la réussite de sa fonction. Il ne s'agit donc pas simplement d'un outil spécifique, mais d'une organisation fonctionnelle de l'emploi de l'outil informatique, qui peut être optimisé par des logiciels non structurants.
\newline
\newline
Par sa nature, le Smart Computing n'est pas techniquement structurant, et n’impacte pas les choix d’architecture de la direction informatique. Il ne l'est pas non-plus de la fonction RH. Sa vocation à libérer l'opérateur des tâches informatiques sans valeur ajoutée, s'exprime dans sa facilité d'utilisation, son absence de besoin en formation, sa caractéristique à s'auto contrôler au service de l'utilisateur.
\newline
\newline
Dès lors il permet un remarquable taux d'acceptabilité des personnels et de leurs organisations représentatives. Derrière l'avancée technique, il est perçu comme une véritable avancée sociale qui valorise le travail utile, conforte la performance et améliore l'implication des personnels. Ce confort dans le travail est apprécié positivement par les syndicats des grandes entreprises qui en ont fait l'expérience.
\newline
\newline
Le Smart Computing est enfin une réponse pertinente aux besoins en RH dans la gouvernance de l'entreprise. Celles-ci sont sous la pression croissante et les exigences pointilleuses des règlementations, des ONG, et des clients. La concurrence accrue et le rythme accéléré des avancées technologiques imposent aux organisations d'augmenter leur capacité à innover et produire avec une agilité maîtrisée.
\newline
\newline
Les prospectives d'investissement des grands comptes placent le Smart Computing comme l'une des 10 technologies porteuses de ces prochaines années. Il conviendrait également de placer cette technologie comme un investissement créateur de valeur dans la ressource humaine, tant ses effets positifs apportent à cette fonction.
\newline
\newline
Etant inscrites dans ce contexte, FY COMPUTING \& iRevolution ont uni leurs compétences pour créer Smart Project, une innovation destinée aux grands groupes qui veulent bénéficier des avantages du Smart Computing. Cette innovation se remarque sur plusieurs points lors de la réalisation d’un projet à savoir :

\subsubsection{Des Nouvelles Façons Motivantes de Travailler}
S’il est une chose dont l’informatique d’entreprise traditionnelle ne manque pas, ce sont les nombreux documents : cahiers des charges, spécifications, dossiers techniques, grilles et des piles de documentation. Tout cela est de nature à rassurer les ingénieurs et les consultants mais ne crée pas la motivation et la stimulation dont les équipes business ont besoin. Et quand en plus, le résultat n’est pas au rendez-vous, la déception est à la hauteur de l’effort consenti. « L’un des aspects les plus merveilleux de Smart Project, c’est qu’il permet d’obtenir des résultats spectaculaires et très remarquables. Auparavant, le premier critère pour juger une marque était la qualité du produit ; désormais, c’est la qualité de l’expérience qui importe avant tout. » explique Grégory Pallière, D.G d’iRévolution.

\subsubsection{Bâtir un Pont entre le Business et la Technologie}
« Smart Project révolutionne non seulement notre façon de gérer les projets, mais aussi la façon de travailler de toute l’équipe », explique Mohamed Feidi BOUZAGBAH. « Pour chaque projet, l’une de nos priorités est de bâtir un pont entre la technologie et le business auquel le projet est destiné. Smart Project nous aide à atteindre cet objectif » ajoute-il, « Pour nous, Smart Project n’est pas juste un moyen de faire des économies ni de travailler plus vite », complète Grégory Pallière. « Il nous aide aussi à améliorer les réponses aux enjeux business à chaque étape de notre travail, ce qui sans cahier des charges, sans spécifications laborieuses, ni réunions fastidieuses ! » déclare Yahya EL MIR.

\subsubsection{Actions Focalisées, Excellence 2.0}
Les actions de FY COMPUTING sont focalisées sur ce qui est le plus important pour les clients. « La simplicité, c’est la sophistication extrême » disait Steve Jobs. Smart Project met un focus tout particulier sur le design et l’expérience utilisateur. « L’expérience utilisateur est la clé aujourd’hui d’un marketing efficace, et l’expression la plus puissante de cela se traduit par un meilleur produit au final.»

\section{Motivation et Problématique du Projet}

Au sein de l’Entreprise, généralement l'équipe de développement des applications est chargée d’exécuter les missions suivantes : Réaliser une étude des besoins fonctionnels pour élaborer un cahier des charges de l’applications à développer, puis concevoir la partie technique du projet pour ensuite la traduire en code source, ce dernier doît être testé avant la phase de déploiement.
\newline

Le code source développé par l’équipe de développement est souvent testé dans plusieurs environnements avant de le mettre en production; en local, puis en environnement de tests, et enfin dans un environnement de pré-production. Si les test dans un environnement donné sont passés avec succès, l'équipe de développement met le code à la disposition de l’équipe de l’exploitation pour l’exploiter dans l’environnement suivant.
\newline

Ce paradigme pose un problème : lorsque les deux équipes travaillent séparément, le développeur peut ne pas être au courant des obstacles opérationnels qui empêchent le programme de fonctionner comme attendu.
\newline

Ce cloisonnement entre les équipes a un impact non négligeable sur le business :
\begin{itemize}
\item Des applications qui ne fonctionnent pas en production malgré les tests.
\item Une durée de déploiement trop importante.
\item Un « Time To Market » trop important.
\item Une impossibilité de livrer rapidement en production un correctif de bug.
\item Des difficultés à gérer l’ensemble des configurations.
\item Des retards de livraisons.
\item Des problèmes de performance des applications.
\item Des difficultés à augmenter rapidement la capacité d’une application.
\newline
\end{itemize}

Alors pour remédier ce problème, on cherche à fusionner le développement et le déploiement au sein d'un exercice plus rationalisé, c’est ainsi que le mot ‘DevOps’ a apparu pour décrire cette fusion.
\newline

Pour réussir un bon DevOps, il faut également des outils technologiques qui interviennent dans les phases de développement des applications informatiques :

\begin{itemize}
\item \textbf {Gestion du développement :} la production du code s’effectue avec des outils propres aux développeurs :
\begin{itemize}
\item un IDE (environnement de développement intégré) comme Eclipse, WebStorm, Android Studio, Visual Studio, …).
\item un framework de développement (AngularJS, Ruby on Rails,  NodeJS, …).
\item un outil de requêtage SQL (SQLDevelopper, Toad, …). 
\end{itemize}

\item \textbf {Gestion du stockage du code :} le code doit être poussé sur dépôt central permettant la mutualisation entre les développeurs d’une même équipe. Des outils comme Git, GitLab, GitHub, Bitbucket, CVS, Subversion ou Mercurial peuvent ainsi être utilisés.


\item \textbf {Gestion de l’intégration continue (CI) :} Le code doit générer automatiquement des builds à l’aide d’un gestionnaire d’intégration continue comme Jenkins\cite{jenkinsEssentials} (fork de Hudson), TeamCity, CruiseControl, ...


\item \textbf {Gestion des tests :} les tests unitaires ont des outils de la famille xUnit comme Junit (monde Java), JSUnit, PyUnit, etc. Les tests métiers disposent d’outils comme Selenium, Behat, etc. D’autres types de tests doivent être effectués comme les tests de sécurité (OWASP, etc.), les tests de performances (JavaMelody, outils d’APM, JMeter, etc.).
\newline

D’autres outils complètent la panoplie et permettent d’augmenter l’automatisation et ainsi d’améliorer la productivité :

\item Les plateformes IaaS (Amazon AWS, Google App Engine, Mirosoft Azure, etc.) qui vont permettre de provisionner en automatiques les machines virtuelles (VMs) nécessaires au fonctionnement de l’application.
\item Les gestionnaires de configuration (Puppet et Chef, Ansible).
\item Les gestionnaires de Conteneurisation (Docker\cite{proDocker}\cite{usingDocker}).
\end{itemize}

\section{Objectifs du Projet et Méthodologie de Travail}

\subsection{Objectifs du Projet}

\subsection{Méthodologie Agile}

\section{Conclusion}
