\documentclass[a4paper,11pt,oneside]{report}
\usepackage[T1]{fontenc}
\usepackage[utf8]{inputenc}
\usepackage[francais]{babel}
\usepackage{graphics}
\usepackage{euscript}
\usepackage{amsmath}
\usepackage{amssymb}
\usepackage{amsfonts}
\usepackage{graphicx}
\usepackage{verbatim}
\usepackage{listings}
\usepackage{array}
\usepackage{placeins}
\usepackage{enumitem}
\usepackage{hyperref}
\usepackage{colortbl}
\usepackage{multirow}
%\usepackage{moreverb}
\usepackage{tikz}

\usepackage{a4wide}
%\documentclass{article}
%\usepackage[demo]{graphicx}
\usepackage{caption}
\usepackage{subcaption}
\author{Badr ELOUIZ}

\begin{document}
\sloppy

\makeatletter
  \begin{titlepage}
  \centering
  
    \begin{figure}[t]
        \begin{tabular}{c c c}
            \multirow{7}{*}{\includegraphics[scale=0.45]{HASS.png}} & &             \multirow{7}{*}{\includegraphics[scale=0.35]{logens.png}}\\
             & & \\
             & \textsc{Université Hassan Premier} & \\
             & \textsc{\'Ecole Nationale des Sciences Appliqu\'ees} & \\
             & \textsc{de Khouribga} & \\
             & & \\
             & & \\
        \end{tabular}
    \end{figure}
        \vfill
    {\LARGE \textbf{M\'EMOIRE DE PROJET DE FIN D'\'ETUDES}}\\
        \vspace{0.5em}
    {\large \textbf{Pour l'obtention du}}\\
        \vspace{0.5em}
    {\LARGE \textbf{Diplôme d'Ingénieur d'\'Etat}}\\
        \vspace{0.5em}
    {\large \textbf{Spécialité}}\\
        \vspace{0.5em}
    {\LARGE \textbf{G\'enie Informatique}}\\
        \vfill
    {\large \textbf{Sujet :}}\\
        \vspace{1em} 
    \begin{tabular}{|c|}
    \hline
     \\
    \ \ \ {\LARGE \textbf{Automatisation du Processus de Développement}}\ \ \ \\
    {\LARGE \textbf{des Applications / DevOps}}\\
     \\
    \hline
    \end{tabular}
    \vfill
         {\large \textbf{Réalisé par :}}\\
        \vspace{0.5em}
    {\large \@author} \\
        \vfill
    \begin{tabular}{c c}
    \multicolumn{2}{c}{{\large \textbf{Encadré par :}}} \\
     & \\
    \ \ \ \ {\large M. Youssef BENKIRAN}\ \ \ \ \ \ \ \ \ & \ \ \ \ \ {\large Said EL KAFHALI} \\
    \ \ \ \ {\large Ingénieur Informatique,}\ \ \ \ \ \ \ \ \ & \ \ \ \ \ {\large Professeur à l'ENSA de Khouribga}\\
    \ \ \ \ {\large FY Computing}\ \ \ \ \ \ \ \ \ & \ \ \ \ \ {\large Université Hassan I, Settat, Maroc}\\
    \end{tabular}
        \vfill
    {\large \textbf{Soutenu le 10 Juillet 2017 devant le jury composé de :}}\\
    \ \newline
        $\begin{array}{l l l}
        \text{M. Imad HAFIDI} & \text{Professeur à l'ENSA de Khouribga} & \text{(Président)} \\
        \text{M. Omar EL BENNAY} & \text{Professeur à l'INSA de Khouribga} & \text{(\'Examinateur)} \\
        \text{M. Said EL KAFHALI} & \text{Professeur à l'ENSA de Khouribga} & \text{(\'Encadrant)} \\
        \end{array}$\\
        \vfill
    {\large \textbf{Année universitaire : 2016/2017}}
  \end{titlepage}

\makeatother

\newpage

\ \newline

\newpage

\begin{flushleft}

\textbf{Remerciements}

Pour commencer, je tiens à remercier particulièrement ma famille pour son soutien et sans qui je ne serai pas là où je suis aujourd’hui. Je remercie ma tutrice Nawal GUERMOUCHE pour sa confiance, ainsi que mon encadrant Said EL KAFHALI pour ses conseils. Merci également à la thésard Ikbel GUIDARA  pour son écoute, sa patience et pour le temps précieux qu'elle m’a consacré à chaque fois que j’en avais besoin. Merci à toute l'équipe pédagogique de l'ENSA qui m'a donner la chance de me former sur le côté technique. Enfin, je tiens à remercier tous ceux qui ont eu un rôle, de loin ou de près, dans la construction de ma personne.

\textbf{Terminologie et notations}

Dans ce rapport les termes techniques de l’approche en temps réel seront en italique. Le gras est employé pour la mise en avant de termes importants. L’utilisation des termes techniques peut être en français ou en anglais. Dans tous les cas, une définition est donnée à chaque terme avant son utilisation.

\end{flushleft}

%\newpage
%\begin{otherlanguage}{arabic}
%\begin{abstract}

%يقوم ميدان تنظيم مجموع عمليات شركة أو مجموعة من الشركات على تقديم الدعائم الأساسية لإدارة فعالة صالحة لكافة الميادين، خصوصا عندما تكون هذه العمليات معقدة و متعلقة بالوقت. لتنفيد هذه العمليات، تعتبر الهندسة المعتمدة على الخدمات من بين الحلول الواعدة.

%هذه الهندسة تعتمد بالخصوص على مفهوم التجريد او مايعرف بالفرنسية بالأبستركسيون، حيث تعبر عن عناصرالعملية بخذمات بسيطة قادرة على تكوين أخرى معقدة. تمكننا هذه الطريقة من التركيز على الخصائص الخذماتية و ليس الوظيفية.

%لتحقيق هذا الهدف يجب الإعتماد على عدد مهم من الخدمات المشاركة و المانحة لخصائص مختلفة، ثم اختيار أفضلها. ففي هذا السياق يندرج العمل المقدم في هذا التقرير.

%هذا التقرير يدرس أولا المقاربة المستعملة لاختيار أحسن تركيبة للخدمات المؤسسة لمجموع عمليات الشركة، ثم يصف مختلف وظائف البرنامج الذي يحقق هذا المبتغى. و في الأخير يوضح المقاربة عن طريق تقديم مثال واقعي يعرف ديناميكية كبيرة.

%\end{abstract}
%\end{otherlanguage}

\newpage

\begin{figure}[p]
\centering
\includegraphics[scale=1]{mol.png}
\end{figure}

\newpage
\begin{abstract}

La discipline BPM (Business Process Management) a pour but de fournir des supports pour une gestion efficace des processus dans différents domaines, tels que le domaine médical et d'entreprise. Ces processus peuvent être très complexes et peuvent être soumis à de fortes contraintes, tel que les contraintes liées au temps. Afin d'implémenter ces processus, l'architecture orientée service (Service Oriented Architecture SOA) semble être une solution prometteuse. En effet, elle repose sur la notion d' encapsulation des applications sous forme de services qui peuvent être composés afin de produire des services plus complexes à valeur ajoutée. 

Afin d'implémenter des processus métiers, plusieurs services offrant les mêmes fonctionnalités mais avec des qualités de services (Quality of Service QoS) différentes peuvent être candidats. Afin d'optimiser ces processus, on doit se baser sur des techniques de sélection.  C'est dans ce contexte que s'inscrit le travail présenté dans ce rapport. 

Ce rapport traitera en premier lieu l’étude et l’implémentation d'une approche de sélection en temps réel (i.e., durant l'execution des processus). En second lieu, la plateforme \textit{BestCombin} de simulation est présentée. Finalement, un cas d’étude réel est proposé.

\end{abstract}

\newpage
\tableofcontents

\newpage

\chapter*{Liste des Figures}

\newpage

\chapter*{Liste des Tableaux}

\newpage

\chapter*{Introduction Générale}

\newpage

\chapter{Présentation de l'Organisme d'Accueil}

\newpage

\section{Les Valeurs de FYComputing}
Parmi les valeurs de l’entreprise FY COMPUTING on trouve : 

\begin{itemize}
\item \textbf{Le goût de l’innovation :}

L’entreprise cherche en permanence à offrir des produits/services uniques sur le marché et dont les clients ne pourront pas s’en passer.


\item \textbf{La culture du partenariat :}

Pour innover, aller vite, satisfaire ses clients finaux, obtenir des résultats, un écosystème de bons partenaires fiables et pointus dans leur domaines est indispensable. Cela oblige à avoir des partenaires qui naturellement ont le goût de l’excellence et du travail bien fait.


\item \textbf{Santé financière :}

FY COMPUTING considère la santé financière d’une entreprise comme la santé physique d’une personne. Vous ne pouvez réellement entreprendre et aller au bout de vos idées sans une bonne santé. C’est la raison pour laquelle elle est dotée d’un positionnement basé sur l’excellence et d’un business model solide, garantissant la bonne santé financière de l’entreprise et son indépendance.
\end{itemize}

\newpage

\chapter{Contexte du Projet et Problématique}

\newpage

\chapter{Technologies Utilisées}

\newpage

\chapter{Réalisation}

\newpage

\chapter*{Conclusion et Perspectives}

\end{document}
